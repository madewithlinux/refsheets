% # (c) Copyright 2015 Josh Wright
\documentclass{article}
\usepackage{mathrsfs,amsmath,amsthm,latexsym,paralist}
\usepackage{mathtools} 
\usepackage{bm} 
\usepackage{amssymb}   % for \varnothing, \therefore
\usepackage{centernot} % for \centernot
\usepackage{geometry}  % for margins
\usepackage{outlines}  % for outline
% \usepackage{paralist}  % for compactitem, compactenum
\usepackage{multicol}
\usepackage{hyperref}
\hypersetup{
	pdftitle={MATH 251 Reference Sheet},
	pdfauthor={Josh Wright},
	pdfsubject={MATH 251},
	bookmarksnumbered=true,
	bookmarksopen=true,
	bookmarksopenlevel=1,
	colorlinks=true,
	pdfstartview=Fit,
	pdfpagemode=UseOutlines,
	colorlinks=true,
	linkcolor=blue,
	filecolor=magenta,      
	urlcolor=cyan,
}
\geometry{letterpaper, margin=0.25in}
\newcommand{\upspace}{\vspace{0px}}
% 
\newcommand{\zza}[1]{\upspace \1 \textbf{#1:} \addcontentsline{toc}{subsubsection}{#1} }
\newcommand{\zzz}[1]{\0 {\textbf{#1:}} \addcontentsline{toc} {subsection}{#1}}
\newcommand{\aaa}{\upspace \1}
\newcommand{\bbb}{\upspace \2}
\newcommand{\ccc}{\upspace \3}
\newcommand{\ddd}{\upspace \4}
% 
% \newcommand{\zzz}[1]{\\ {\large\textbf{#1:}} \addcontentsline{toc} {subsection}{#1}}
% \newcommand{\aaa}{\upspace \\ \noindent}
% \newcommand{\bbb}{\upspace \\ \indent}
% \newcommand{\ccc}{\upspace \\ \indent \indent}
% \newcommand{\ddd}{\upspace \\ \indent \indent \indent}

% makes second-level itemize bullets instead of dashes
% \renewcommand\labelitemi{\cdot}
\renewcommand\labelitemi{\tiny$\bullet$}
\renewcommand\labelitemii{\labelitemi}

\newcommand{\p}{\partial}
\newcommand{\ang}[1]{\left\langle #1 \right\rangle}
\newcommand{\ceil}[1]{\left\lceil #1 \right\rceil}
\newcommand{\floor}[1]{\left\lfloor #1 \right\rfloor}
\newcommand{\prt}[2]{\frac{\partial#1}{\partial#2}}
\newcommand{\grad}{\nabla}
	
\begin{document}
\allowdisplaybreaks
% \begin{center}
\noindent
\textbf{Math 251 Reference Sheet} \hfill \textcopyright \space Josh Wright 2015 \\
Latex Symbols: \url{http://oeis.org/wiki/List_of_LaTeX_mathematical_symbols} \hfill Last Updated: \today \\
Paul's Online Math Notes: \url{http://tutorial.math.lamar.edu/Classes/CalcIII/CalcIII.aspx} \\
% \end{center}
% asterisk makes multicols finish one column before going onto the next
\begin{multicols*}{2}
% \tableofcontents
% \large
\begin{outline}[compactitem]
\noindent
\zzz{3D vectors}
	\aaa vector $v = \langle a,b,c \rangle$
	\aaa Magnitude (length): $|v| = \sqrt{a^2 + b^2 + c^2}$=
	\aaa \textbf{Dot Product:} $a \cdot b = a_1b_1 + a_2b_2 + a_3b_3 = |a||b|\cos{\theta}$
	\\ if $a \cdot b = 0$, $a$ and $b$ are perpendicular
	\aaa \textbf{Cross Product:} $a \times b =\langle a_yb_z-a_zb_y,a_zb_x-a_xb_z,a_xb_y-a_yb_x \rangle$
	\\ if $a \times b = 0$, $a$ and $b$ are parallel
	\\ $a \times b = n|a||b|\sin{\theta}$ where $n$ is a vector perpendicular to both $a$ and $b$ in direction given by right hand rule
	\\ $a\times (b+c) = a\times b + a\times c$
	\aaa \textbf{Angle} between (nonzero) vectors: $\theta = \cos^{-1}(\frac{a\cdot b}{|a||b|})$
	\aaa Unit Vector: $\hat{a} = \frac{a}{|a|}$
	\\ $\hat{a}$ is a vector of length 1 parallel to vector $a$
	\aaa Scalar triple product: $a \cdot (b \times c)$
	\aaa Vector triple product: $a \times (b \times c)$
	\aaa areas and volumes:
		\bbb area of parallelogram with sides $a,b = |a \times b|$
		\bbb area of triangle with sides $a,b = \frac{1}{2} |a \times b|$
		\bbb volume of box with sides $a,b,c = a \cdot (b \times c)$
\zzz {Lines}
	\aaa \textbf{Vector Equation:} $L(t) = r_0 + vt$
	\\ $r_0$ is a point on the line and $v$ is a vector parallel to the line
	\\ $L(t) = \langle x_0,y_0,z_0 \rangle + t\langle a,b,c \rangle$
	\aaa \textbf{Parametric Equation:} $L(t) = \langle x_0 + at, y_0 + bt, z_0 + ct \rangle$
	\\ point on line: $(x_0,y_0,z_0)$
	\\ vector parallel to line: $\langle a,b,c \rangle$
\zzz{Planes}
	\aaa \textbf{Standard (linear) form:} $ax + by + cz = d$
	\\ $d = ax_0 + by_0 + cz_0 $ where $P(x_0,y_0,z_0)$ is a point in the plane
	\\ normal vector: $n = \langle a,b,c \rangle$
	\aaa \textbf{Scalar form:} $a(x-x_0) + b(y-y_0) + c(z-z_0) = 0$
	\\ normal vector: $n = \langle a,b,c \rangle$
	\\ point in plane: $P(x_0,y_0,z_0)$
	\aaa \textbf{Distance} from point $P(x,y,z)$ to plane: $D = \frac{|ax + by + cz - d|}{\sqrt{a^2 + b^2 + c^2}}$
	\\ (assuming plane is in linear form above)
\zzz{Quadratic Surfaces}
	\aaa \textbf{Ellipsoid}                    \dotfill $\frac{x^2}{a^2} + \frac{y^2}{b^2} + \frac{z^2}{c^2} = 1$
		\bbb \textbf{All} traces are ellipses
	\aaa \textbf{Elliptic Paraboloid}          \dotfill $\frac{x^2}{a^2} + \frac{y^2}{b^2} = \frac{z}{c}$
		\bbb \textbf{Horizontal} traces are ellipses
		\bbb \textbf{Vertical}   traces are parabolas
	\aaa \textbf{Hyperboloid} of one sheet     \dotfill $\frac{x^2}{a^2} + \frac{y^2}{b^2} - \frac{z^2}{c^2} = 1$
		\bbb \textbf{Horizontal} traces are ellipses
		\bbb \textbf{Vertical}   traces are hyperbolas
	\aaa \textbf{Hyperboloid} of two sheets    \dotfill $ -\frac{x^2}{a^2} - \frac{y^2}{b^2} + \frac{z^2}{c^2} = 1$
		\bbb \textbf{Horizontal} traces are ellipses
		\bbb \textbf{Vertical}   traces are hyperbolas
		\bbb some traces do not exist because graph has a gap centered around the origin
	\aaa \textbf{Cone}                         \dotfill $\frac{x^2}{a^2} + \frac{y^2}{b^2} = \frac{z^2}{c^2}$
		\bbb \textbf{Horizontal} traces are ellipses.
		\bbb \textbf{Vertical}   traces are pair of lines if $x$ or $y$ is 0, otherwise hyperbolas
	\aaa \textbf{Hyperbolic Paraboloid}        \dotfill $\frac{x^2}{a^2} - \frac{y^2}{b^2} = \frac{z}{c}$
		\bbb \textbf{Horizontal} traces are hyperbolas.
		\bbb \textbf{Vertical}   traces are parabolas
\zzz{Vector Functions}
	\aaa \textbf{Arc length} from $t=a$ to $t=b$: $\int_a^b |r'(t)| dt$
		\bbb Conversion back to the similar form from 2D:
		\bbb $|r'(t)| = \sqrt{(f_x)^2 + (f_y)^2 + (f_z)^2}$
	\aaa \textbf{Arc Length Function:} $s(t) = \int_a^t |r'(u)| du$
	\aaa \textbf{Unit Tangent Vector:} $T(t) = \frac{r'(t)}{|r'(t)|}$
	\\ unit-length vector tangent to the curve $r(t)$
	\aaa \textbf{Unit normal vector:} $N(t) = \frac{T'(t)}{|T'(t)|}$
	\\ unit-length vector perpendicular to $r(t)$
\zzz{Derivatives}
	\aaa $z = f(x,y)$
	\aaa Notation: $\frac{\partial z}{\partial x} = f_x = \frac{\partial f}{\partial x} $ etc\ldots
		\bbb Same for second derivatives: $f_{xy} = \frac{}{}$
	\aaa \textbf{Gradient Vector:} $\nabla f = \langle f_x, f_y, f_z \rangle$
	\aaa \textbf{Tangent plane} at $P(a,b,c)$: $f_x(a,b) + f_y(a,b) = z - c$
	\aaa Chain Rule:
		\bbb $\frac{\partial x}{\partial z} = -\frac{\partial F / \partial z }{\partial F / \partial x} = \frac{F_z}{F_x}$ ($\partial F$ cancels out, fraction flips)
		\bbb $\frac{\partial z}{\partial t} = -\frac{\partial z}{\partial x} \frac{\partial x}{\partial t}$ ($\partial x$ cancels out)
	\aaa Directional Derivative, parallel to $\langle a,b,c \rangle$ at $P(x,y,z)$: $f_x(x,y,z)a + f_y(x,y,z)b + f_z(x,y,z)c$
\zzz{Double Integrals}
	\aaa volume under the function $f(x,y)$ on the rectangle $R = [a,b]\times[c,d]$
	\aaa $\iint_R f(x,y) dA$ on $R = [a,b]\times[c,d] = \int_c^d\int_a^bf(x,y)dxdy$
	\\ solve the inner integral first, then the outer
	\\ if $f(x,y)$ is continuous on $R$, then you can flip the order of the integrals
	\aaa if $f(x,y)=g(x)\cdot h(y)$ then $\int_c^d\int_a^bf(x,y)dxdy=\int_a^bg(x)dx\cdot\int_c^dh(y)dy$
	\aaa \textbf{General Regions:} 
	\\ Only difference is whether $x$ or $y$ has its limits defined in terms of the other
	\\ For these, you must evaluate the inner integral first, you can't swap them
		\bbb Type 1: bounds of $x$ are constants, bounds of $y$ are defined as functions of $x$
		\\ $D = \{(x,y)|a\leq x\leq b, g_1(x)\leq y\leq g_2(x)\}$
		\\ $=\int_a^b \int_{g_1(x)}^{g_2(x)} f(x,y) dy dx $
		\bbb Type 2: bounds of $y$ are constants, bounds of $x$ are defined as functions of $y$
		\\ $D = \{(x,y)|h_1(y)\leq x\leq h_2(y),c\leq y\leq d\}$
		\\ $=\int_c^d \int_{h_1(y)}^{h_2(y)} f(x,y) dx dy $
\zzz{Polar Coordinates}
	\aaa Polar $\rightarrow$ Cartesian:
	\\ $r = \pm\sqrt{x^2 + y^2}$
	\\ $\theta = \tan^{-1}(\frac{y}{x})$
		\bbb You need to be careful with the sign of $r$ and multiples of $\theta$ because Polar Coordinates are \textbf{not} unique.
	\aaa Cartesian $\rightarrow$ Polar:
	\\ $x = r\cos\theta$
	\\ $y = r\sin\theta$
		\bbb You don't have to worry about quadrants or anything because Cartesian Coordinates \textbf{are} unique.
	\aaa \textbf{Double Integrals in Polar Coordinates}
	\\ works best when D is in a polar-coordinate-friendly shape
		\bbb Do the following replacements:
			\ccc $dA$ or $dxdy$ or $dydx$ $\rightarrow r dr d\theta$
			\ccc $x \rightarrow r\cos\theta$
			\ccc $y \rightarrow r\sin\theta$
			\ccc $x^2 + y^2 \rightarrow r$
			\ccc Translate limits
		\bbb Should end up with something that looks like one of these general regions:
		\\ $\int_a^b \int_{g_1(r)}^{g_2(r)} f(r\cos\theta,r\sin\theta) dx dy $
		\\ $\int_\alpha^\beta \int_{h_1(\theta)}^{h_2(\theta)} f(r\cos\theta,r\sin\theta) dr d\theta $
		\bbb integrate as normal with new function and new limits
		\\ stuff will probably cancel out everywhere
	\aaa \textbf{Arc Length in Polar Coordinates:}
	\\ $L = \int_{\theta=\alpha}^{\theta=\beta}\sqrt{ \left(\frac{dx}{d\theta} \right)^2 + \left(\frac{dy}{d\theta}\right)^2} d\theta = \int_{\theta=\alpha}^{\theta=\beta}\sqrt{ r^2 + \left( \frac{dr}{d\theta} \right)^2} d\theta$
\\
\zzz{Cylindrical Coordinates}
	\aaa $dV = r dr d\theta dz$
	\aaa usually:
		\\ $r \geq 0$
		\\ $0 \leq \theta \leq 2\pi$
	\aaa Cylindrical $\rightarrow$ Cartesian:
	\\ $x = r\cos\theta$
	\\ $y = r\sin\theta$
	\\ $z = z$
	\aaa Cartesian $\rightarrow$ Cylindrical:
	\\ (be careful about the quadrant of $\theta$)
	\\ $r = \sqrt{x^2 + y^2}$
	\\ $r^2 = x^2 + y^2$
	\\ $\theta = \tan^{-1}\left(\frac{y}{x}\right)$
	\\ $z = z$
\\
\zzz{Spherical Coordinates}
	\aaa $dV = \rho^2 \sin\phi$
	\aaa $\phi$ is the angle from the $+z$ axis down to $\rho$
		\bbb usually:
		\\ $-\pi/2 \leq \rho \leq \pi/2$
		\\ $\rho \geq 0$
		\\ $0 \leq \theta \leq 2\pi$
	\aaa Spherical $\rightarrow$ Cartesian:
	\\ $x = \rho\sin\phi\cos\theta$
	\\ $y = \rho\sin\phi\sin\theta$
	\\ $z = \rho\cos\phi$
	\aaa Cartesian $\rightarrow$ Spherical:
	\\ $\theta = \tan^{-1}\left(\frac{y}{x}\right)$
	\\ $\rho^2 = x^2 + y^2 + z^2$
	\\ $\phi = \cos^{-1}\left(\frac{z}{\rho}\right)$
	\aaa Other:
	\\ $\rho^2 = x^2 + y^2 + z^2$
	\\ $r = \rho \sin\phi$
\\
\zzz{Minimum and Maximum}
\\ Find critical points by solving $\grad f = \ang{0,0}$
\\ For each point, find $D = (f_{xx})(f_{yy}) - \left(f_{xy}\right)^2$
\\ $D>0$ and $f_{xx}>0$: relative minimum at $(a,b)$
\\ $D>0$ and $f_{xx}<0$: relative maximum at $(a,b)$
\\ $D<0$: saddle point at $(a,b)$
\\ $D=0$: can't tell (probably won't see)
\\
\\
\zzz{Line Integrals}
	\aaa\textbf{2D:}
	\\ $C = \{r(t) | a \leq r \leq b\}$
	\\ $r(t) = \ang{x,y,z}$
	\\ Scalar function $f(x,y)$:
	\\ $\int_C f(x,y) ds = \int_a^b f(r(t)) \sqrt{(x'(t))^2 + (y'(t))^2} dt$
	\\ $= \int_a^b f(r(t))|r'(t)| dt$
	\\ $\int_C f(x,y)dx = \int_a^b f(r(t))x'(t)dt$
	\\ Vector field $F(x,y) = \langle P, Q \rangle$:
	\\ $\int_C F(x,y) \cdot ds = \int_b^a P(r(t)) dx + Q(r(t)) dy $
	\aaa\textbf{3D:}
	\\ $C = \{r(t) | a \leq r \leq b\}$
	\\ (scalars are the same)
	\\ Vector field $F(x,y,z) = \langle P, Q, R \rangle$:
	\\ $\int_C F(x,y,z) \cdot ds  = \int_a^b F(r(t))\cdot r'(t) dt$
	\\ $= \int_C P dx + Q dy + R dy $
	\\ $= \int_a^b P\prt{r}{x} dx + Q\prt{r}{y} dy + R\prt{r}{z} dz $
	\\ $\int_C F(x,y,z) \cdot dr = \int_a^b F(r(t)) \cdot r'(t) dt $
\\
\\
\zzz{Fundamental Theorem for Line Integrals}
\\ $C: r(t), a\leq r \leq b$, $C$ is simple. Domain is simply-connected
\\ $F(x,y,z) = \ang{P,Q,R}$
\\ If there exists $f$ such that $\grad f = F$, then:
\\ $\int_C F \cdot dr = f(r(b)) - f(r(a))$
\\
\\
\zzz{Green's Theorem} (2D only, doesn't work in 3D)
\\ $C$: curve with $r(t) = \ang{x(t),y(t),z(t)} $ on $a\leq t\leq b$; $C$ is closed and simple; $D$: region enclosed by $C$
\\ $F(x,y) = \ang{P(x,y),Q(x,y)}$
\\ $\int_C F\cdot dr = \int_C Pdx + Qdy = \iint_D \left(\prt{Q}{x} - \prt{P}{y} \right) dA $
\\\zzz{Vector Field: Curl and Divergence}
\\ $curlF = \nabla\times F = \ang{\prt{}{x}, \prt{}{y}, \prt{}{z}}\times\ang{P,Q,R}$
\\ $= \ang{R_y-Q_z, P_z-R_x, Q_x-P_y}$
\\ $divF = \nabla\cdot F =  \prt{P}{x}+\prt{Q}{y}+\prt{R}{z} = P_x+Q_y+R_z$
\\ $div(curl(F)) = 0, curl(div(f)) = 0$
\\ if $curl(F) = 0$, then $F$ is irrotational (causes no rotation); and therefore $\nabla f = F$ exists
\\
\\
\zzz{Surface Integrals}
\\ Surface $S: r(u,v)=\ang{x(u,v), y(u,v), z(u,v)}$ for $(u,v) \in D$
\\ scalar function $f(x,y,z)$, vector field $F(x,y,z)$
\\ $\hat{n} = (r_u\times r_v)/|r_u\times r_v| = \frac{r_u\times r_v}{|r_u\times r_v|}$
\\ Scalar Function $f$:
\\ $\iint_S f dS = \iint_D f(r(u,v))|r_u\times r_v|dA$
\\ When $f(x,y,z): z=g(x,y)$: 
	\\ $\iint_S f(x,y,z)dS = \iint_D f(x,y,g(x,y))\sqrt{g_x^2 + g_y^2 + 1}dA$ 
\\ Area of $S: \iint_D |r_u\times r_v|dA$
\\ Vector Function $F$:
\\ $\iint_S F \cdot dS = \iint_S (F\cdot\hat{n})dS = \iint_D F(r(u,v))\cdot(r_u\times r_v) dA$
\\
\\
\zzz{Stokes' Theorem} (works in 3D)
\\ Surface $S$ bounded by curve $C: g(t), a\leq t \leq b$
\\ vector field $F(x,y,z)$
\\ $\int_C F \cdot dg = \iint_S curl(F)\cdot dS$
\\
\\
\zzz{Divergence Theorem}
\\ surface $S$ is boundary of solid region $E$
\\ $F$ is vector field
\\ $\iint_S F\cdot dS = \iiint_E div(F) dV$
\\
\\
test this stuff like editing in emacs vim spacemacs

\end{outline}
\end{multicols*}
\end{document}
