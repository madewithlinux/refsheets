% # (c) Copyright 2015 Josh Wright
\documentclass{article}
%%%%%%%%%%%%%%
%% packages %%
%%%%%%%%%%%%%%
\usepackage{mathrsfs,amsmath,amsthm,latexsym,paralist}
\usepackage{mathtools} 
\usepackage{bm}
\usepackage{calc}      % for dimension arithmetic
\usepackage{amssymb}   % for \varnothing, \therefore
\usepackage{centernot} % for \centernot
\usepackage{geometry}  % for margins
\usepackage{outlines}  % for outline
% \usepackage{paralist}  % for compactitem, compactenum
\usepackage{multicol}
\usepackage{hyperref}
%%%%%%%%%%%%%%%%
%% hyperlinks %%
%%%%%%%%%%%%%%%%
\hypersetup{
  pdftitle={ECEN 214 Reference Sheet},
  pdfauthor={Josh Wright},
  pdfsubject={ECEN 214},
  bookmarksnumbered=true,
  bookmarksopen=true,
  bookmarksopenlevel=1,
  colorlinks=true,
  pdfstartview=Fit,
  pdfpagemode=UseOutlines,
  colorlinks=true,
  linkcolor=blue,
  filecolor=magenta,      
  urlcolor=cyan,
}
%%%%%%%%%%%%%%%%%%%%%%%%%%%%%%%%%%%%%%
%% paper size, orientation, margins %%
%%%%%%%%%%%%%%%%%%%%%%%%%%%%%%%%%%%%%%
% good for on screen-only viewing
\def \columncount {3}
\newlength{\papersize}
\setlength{\papersize}{18cm}
\geometry{paperheight=1.7777\papersize, paperwidth=\papersize, landscape, margin=0.25in}
% good for printing
% \def \columncount {2}
% \geometry{letterpaper, portrait, margin=0.25in}

% makes second-level itemize bullets instead of dashes
% \renewcommand\labelitemi{\cdot}
\renewcommand\labelitemi{\tiny$\bullet$}
\renewcommand\labelitemii{\labelitemi}

%%%%%%%%%%%%%%%%%%%%
%% helpful macros %%
%%%%%%%%%%%%%%%%%%%%
\newcommand{\p}{\partial}
\newcommand{\ang}[1]{\left\langle #1 \right\rangle}
\newcommand{\ceil}[1]{\left\lceil #1 \right\rceil}
\newcommand{\floor}[1]{\left\lfloor #1 \right\rfloor}
\newcommand{\prt}[2]{\frac{\partial#1}{\partial#2}}
\newcommand{\diff}[2]{\frac{d #1}{d #2}}
\newcommand{\grad}{\nabla}

\begin{document}
\allowdisplaybreaks
\large
% \Large
%%%%%%%%%%%
%% title %%
%%%%%%%%%%%
\noindent
\textbf{ECEN 214 Reference Sheet} \hfill Last Updated: \today \hfill \textcopyright \space Josh Wright
\\ Latex Symbols: \url{http://oeis.org/wiki/List_of_LaTeX_mathematical_symbols}
\begin{multicols*}{\columncount}
\begin{outline}[compactitem]

%%%%%%%%%%%%%%%%%%%%
%% spacing config %%
%%%%%%%%%%%%%%%%%%%%
% just in case I need even more space
\newcommand{\upspace}{\vspace{0px}\linespread{0}}
% section titles
\newcommand{\zzz}[1]{\noindent\0\noindent {\textbf{#1:}}}
% makes second-level itemize bullets instead of dashes
\renewcommand\labelitemii{\labelitemi}
% redefine the sub-headings to inject our space-saver
\let\oldOne\1\let\oldTwo\2\let\oldThree\3\let\oldFour\4
\renewcommand{\1}{\upspace \oldOne  }
\renewcommand{\2}{\upspace \oldTwo  }
\renewcommand{\3}{\upspace \oldThree}
\renewcommand{\4}{\upspace \oldFour }




%%%%%%%%%%%%%%%%%%%%%%%%%%%%%%
%% real content starts here %%
%%%%%%%%%%%%%%%%%%%%%%%%%%%%%%

\zzz{Metric Prefixes} \\
\begin{tabular}{|c c l l|}                                   \hline
peta  & P     & $10^{ 15}$ & \hfill 1 000 000 000 000 000 \\ \hline
tera  & T     & $10^{ 12}$ & \hfill     1 000 000 000 000 \\ \hline
giga  & G     & $10^{  9}$ & \hfill         1 000 000 000 \\ \hline
mega  & M     & $10^{  6}$ & \hfill             1 000 000 \\ \hline
kilo  & k     & $10^{  3}$ & \hfill                 1 000 \\ \hline
hecto & h     & $10^{  2}$ & \hfill                   100 \\ \hline
deca  & da    & $10^{  1}$ & \hfill                    10 \\ \hline
one   &       & $10^{ 0 }$ & \hfill       1 \hfill \hfill \\ \hline
deci  & d     & $10^{- 1}$ & 0.1                          \\ \hline
centi & c     & $10^{- 2}$ & 0.01                         \\ \hline
milli & m     & $10^{- 3}$ & 0.001                        \\ \hline
micro & $\mu$ & $10^{- 6}$ & 0.000 001                    \\ \hline
nano  & n     & $10^{- 9}$ & 0.000 000 001                \\ \hline
pico  & p     & $10^{-12}$ & 0.000 000 000 001            \\ \hline
femto & f     & $10^{-15}$ & 0.000 000 000 000 001        \\ \hline
\end{tabular}

\zzz{Ohm's Law}
  $V = IR$,
  \quad $I = \frac{V}{R}$,
  \quad $R = \frac{V}{I}$

\zzz{Power} 
  $P = IV = I^2 R = \frac{V^2}{R}$

\zzz{Energy}
  \1 $W = \int_{0}^{t}P(s) ds$
  \1 Unit: Watts, $W=\frac{J}{s}=\frac{V^2}{\Omega}=VA=A^2\Omega$

\zzz{KCL: Kirchoff's Current Law}
  \1 All currents out of (or into) a point sum to 0
  \1 be careful with signs with this!

\zzz{KVL: Kirchoff's Voltage Law}
  \1 The sum of voltages around a fixed loop is 0 
  \1 be careful with signs with this one too!




\zzz{\Large Exam 1 stuff}

\zzz{Source Transformation}
  \1 When you have a current source with a resistor in parallel to its load or a voltage source with a resistor in series to it's load, you can use Ohm's Law to transform it into the opposite source type.
  \1 The resistor value stays constant, the source type and value changes

\zzz{Nodal Analysis}
  \1 use KCL to sum all currents at each node to 0
  \1 when in doubt, use more nodes
  \1 remember to have a ground node for reference
  \1 convention is to count current out of the node as positive
  \2 current sources pointing into the node are counted negative

\zzz{Superposition}
  \1 evaluate the circuit many times, killing all but one source each time
  \1 sum the results to get the final result (be careful with signs)
  \1 superposition only applies for linear circuits
    \2 which nearly all are. Notable exception: stuff with diodes

\zzz{Mesh}(current loop method)
  \1 use KVL to sum all the voltages in each loop to 0
  \1 convention is to loop clockwise
  \1 when you hit a voltage terminal, use the sign of that terminal; e.g. hit negative terminal of $V_x$ means append ``$-V_x$'' to equation. (does not matter if $V_s$ is source or component)
  \1 When you hit a resistor, use $V=IR$ to find voltage drop

\zzz{Thevenin: Independent Only}
  \1 always remove the load resistor (if present) first
  \1 find $V_{th}$ by assuming $AB$ is an open circuit
  \1 find $R_{th}$:
    \2 first deactivate all sources in the circuit
    \2 then determine the equivalent resistance from $A$ to $B$

\zzz{Thevenin: Dependent and Independent}
  \1 Note: you can't kill independent sources
  \1 find $V_{th}$ the same way
  \1 find $R_{th}$:
    \2 kill all independent sources
    \2 put a 1Amp independent current source across $AB$
    \2 find voltage across new independent source
    \2 use $R_{th}=\frac{V}{I}=\frac{V}{1\mbox{\small Amp}}$

\zzz{Thevenin: Dependent Only}
  \1 can't find $V_{th}$ regular way, so use $1$Amp source method
  \1 This means you will get $R_{th}=\frac{V_{th}}{1\mbox{\small Amp}}$

\zzz{Norton Equivalent}
  \1 just use Ohm's Law on the Thevenin Equivalent

\zzz{\Large Exam 2 stuff}

\zzz{Ideal Op Amp}
  \1 The output tries to do whatever is necessary to make the difference between the input voltages zero.
  \1 Zero current in/out from the input pin
  \1 For a real op amp, the output voltage is limited to $\pm V_{CC}$
    \2 Saturated when $|V_{out}|=V_{CC}$

\zzz{Op Amp - Inverting}
  \1 $V_o \rightarrow R_f \rightarrow$ $-$input
    \\ $-$input $\rightarrow R_s \rightarrow V_s$
    \\ $+$input $\rightarrow$ ground
  \1 $R_f$: feedback resistor
    \\ $R_s$: source resistor
  \1 $V_o = -\frac{R_f}{R_s}V_s$
  \1 linear region: $\left|\frac{R_f}{R_s}\right|\leq\left|\frac{V_{CC}}{V_s}\right|$

\zzz{Op Amp - Summing}
  \1 Adds voltages
  \1 Similar to inverting op amp except that each input is wired as $V_x\rightarrow R_x \rightarrow-$input. (Only one $R_f$, but one $R_s=R_x$ for each input)
  \1 $V_o = -\left(\frac{R_f}{R_a}V_a + \frac{R_f}{R_b}V_b + \ldots \right)$
  \1 If $R_f=R_a=R_b\ldots$ then $V_o=-(V_a + V_b+\ldots)$

\zzz{Op Amp - Non-inverting}
  \1 $+$input$\rightarrow R_s\rightarrow V_g \rightarrow$ground
    \\ $-$input$\rightarrow R_s\rightarrow$ground
    \\ (two different resistors each with value $R_s$)
    \\ same $R_f$ feedback resistor as inverting amp
    \\ $R_f$ and $R_s$ form an unloaded voltage divider across  $-$input
  \1 $V_o = \frac{R_s+R_f}{R_s}V_g$
  \1 Linear region: $\frac{R_s+R_f}{R_s}<\left|\frac{V_{CC}}{V_g}\right|$

\zzz{Op Amp - Difference}
  \1 regular feedback resistor $R_b$
    \\ $R_a$: from $V_a$ to $-$input
    \\ $R_c$: from $V_b$ to $+$input
    \\ $R_d$: from $+$input to ground
  \1 $V_o = \frac{R_d(R_a+R_b)}{R_a(R_c+R_d)}V_b - \frac{R_b}{R_a} V_a$
  \1 if $\frac{R_a}{R_b}=\frac{R_c}{R_d}$ then $V_o=\frac{R_b}{R_a}(V_b-V_a)$

\zzz{Op Amp - Integrator}
  \1 Circuit: just like inverting op-amp except with a capacitor instead of the feedback resistor ($R_f$)
  \1 The only resistor is $R_s$, between the source and the $-$input of the op amp
  \1 $V_o=-\frac{1}{R_sC}\int V_{in} dt$
  \1 can be used to turn square wave into sawtooth wave

\zzz{Op Amp - Differentiator}
  \1 Circuit: just like inverting op-amp except with a capacitor instead of the input resistor ($R_s$)
  \1 $V_o = -R_fC\diff{V_{in}}{t}$
  \1 Can be used to turn sawtooth wave into square wave


\zzz{Inductors}
  \1 Series and parallel is the same as resistors
  \1 Voltage: $V_L(t) = -L\diff{i_L(t)}{t}$
    \\ Energy: $W(t) = \frac{1}{2} L*i(t)^2$
  \1 $\tau = \frac{L}{R}$
  \1 RL Charging: $i(t) = i_f + (i_o - i_f)e^{-tR/L}$
  \1 RL Discharging: $i(t) = i_oe^{-tR/L}$   

\zzz{Capacitors}
  \1 Series and parallel is the opposite as resistors
  \1 Current: $I_c(t) = C \diff{V_C(t)}{t}$
    \\ Energy: $W(t) = \frac{1}{2}C*V_C(t)^2$
  \1 $\tau = RC$
    \\ $V_f=$final voltage
    \\ $V_o=$initial voltage
  \1 RC Charging: $V(t) = V_f + (V_o - V_f)e^{-t/(RC)}$
  \1 RC Discharging: $V(t) = V_oe^{-t/(RC)}$



\zzz{Exam 3 stuff}

\zzz{RLC Circuits - Parallel}
  \1 $s_{1,2} = \frac{1}{2/RC} \pm \sqrt{\left(\frac{1}{2RC}\right)^2-\frac{1}{LC}}$
  \\ $s_{1,2} = - \alpha \pm \sqrt{\alpha^2-\omega_0^2}$
  \1 $\alpha = \frac{1}{2RC}$
  \1 resonant frequency: $\omega_0 = \frac{1}{\sqrt{LC}}$
  \1 damping:
    \2 $\omega_0 < \alpha$: over damped
      \3 $V(t) = A_1 e^{s_1 t} + A_2 e^{s_2 t}$
    \2 $\omega_0 = \alpha$: critically damped
      \3 $V(t) = B_1 t e^{\alpha t} + B_2 e^{\alpha t}$
    \2 $\omega_0 > \alpha$: under damped
      \3 $V(t) = B_1 t e^{\alpha t}\cos(\omega_0t) + B_2 e^{\alpha t}\sin(\omega_0t)$
  \1 $\omega_d = \sqrt{\omega_0^2 - \alpha^2}$ damped frequency (only relevant for under damped)

\zzz{Impedance}
  \1 $V = IZ$
  \1 $Z_R = R$
    \2 $\phi_R=0^\circ$
  \1 $Z_L = j\omega L$
    \2 Current lags $90^\circ$ behind voltage
    \2 $\phi_L=-90^\circ$
  \1 $Z_C = \frac{1}{j\omega C}$
    \2 Current leads voltage by $90^\circ$
    \2 $\phi_C=90^\circ$

\zzz{Reactance}
  \1 $X_L = \frac{V_L}{I_L} = \omega L = 2\pi f L$
  \1 $X_C = -\frac{1}{\omega C} = -\frac{1}{2\pi f C}$


\zzz{Complex Power}
  \1 $S = P + jQ = \frac{(V_{RMS})^2}{Z^*}$
    \2 $\bar{Z}$ is complex conjugate of $Z$
  \1 $P = \frac{V_m I_m}{2}\cos(\theta_v - \theta_i)$
  \1 $Q = \frac{V_m I_m}{2}\sin(\theta_v - \theta_i)$
  \1 $S = V I^*$
    \2 $^*$ also means conjugate

\zzz{Maximum Power Transfer}
  \1 unrestricted: $Z_{L} = \bar{Z}_{Th}$
  \1 restricted: $|Z_{L}| = |\bar{Z}_{Th}|$





\end{outline}
\end{multicols*}
\end{document}
